\documentclass[aspectratio=43]{beamer}
\usepackage{ragged2e}
\usepackage{multirow}
\usepackage{alltt}

\usetheme{CSCS}


\newcommand{\SummerSchoolYear}{2016}
\newcommand{\SummerSchoolDate}{July 20 -- 21}
\newcommand{\SummerSchoolAuthor}{Maxime Martinasso}

\newcommand{\footlinetext}{Summer School \SummerSchoolYear{} -- MPI}

\author{\SummerSchoolAuthor, CSCS}
\title{Message Passing Interface (MPI)}
\subtitle{Summer School \SummerSchoolYear{}  -- Effective High Performance Computing}
\date{\SummerSchoolDate, \SummerSchoolYear}


% Select the image for the title page
\newcommand{\picturetitle}{cscs_images/image3.pdf}
%\newcommand{\picturetitle}{cscs_images/image5.pdf}
%\newcommand{\picturetitle}{cscs_images/image6.pdf}

\begin{document}

% TITLE SLIDE
\cscstitle

\begin{frame}{Course Objectives}
\begin{itemize}
\item Make the MiniApp runs on multiple nodes
\item Add MPI function calls in the MiniApp code
\end{itemize}
\end{frame}

% TABLE OF CONTENT SLIDE
\cscstableofcontents[hideallsubsections]{MiniApp changes}

% CHAPTER SLIDE
\cscschapter{MiniApp changes review}

\section{Initialize and finalize MPI}
\begin{frame}[fragile]{Initialize and finalize MPI}
Setup the code to be MPI aware.
\begin{itemize}
\item File to edit: \lstinlineCpp{main.cu}
\item Initialize MPI and get the current rank and the number of ranks
\item Finalize MPI
\end{itemize}
\end{frame}

\section{Create a Cartesian topology}
\begin{frame}[fragile]{Create a Cartesian topology}
Minapp uses a 2D grid, each rank will work on a sub part of the grid.\\
Make a 2D domain decomposition of the grid depending on the number of ranks.\\
\begin{itemize}
\item File to edit: \lstinlineCpp{data.cu}
\item Create the dimension of the decomposition depending on the number of ranks
\item Create a non-periodic Cartesian topology for the grid of domains
\item Identify coordinates of the current rank in the domain grid
\item Identify neighbours of the current rank: east, west, north and south directions
\end{itemize}
\end{frame}

\section{Change linear algebra functions}
\begin{frame}[fragile]{Change linear algebra functions}
Make the dot product and the computation of the norm over all ranks.
\begin{itemize}
\item File to edit: \lstinlineCpp{linalg.cu}
\item Add a collective operation to compute the dot product
\item Add a collective operation to compute the norm
\end{itemize}
\end{frame}

\section{Exchange ghost cells}
\begin{frame}[fragile]{Exchange ghost cells}
Use point to point communication to exchange ghost cells among neighbours.
Use RDMA.
\begin{itemize}
\item File to edit: \lstinlineCpp{operators.cu}
\item Add point-to-point communication for all neighbours in all directions
\item Use Non-blocking communication
\item Try to overlap computation and communication
\item Optional - try MPI 1-sided
\end{itemize}
\end{frame}

\section{Summary}
\begin{frame}[fragile]{Summary}
\begin{itemize}
\item Initialize MPI:\@ \lstinlineCpp{main.cu}
\item Domain decomposition: \lstinlineCpp{data.cu}
\item Parallel linear algebra: \lstinlineCpp{linalg.cu}
\item Exchange ghost cells: \lstinlineCpp{operators.cu}
\end{itemize}
\end{frame}

% THANK YOU SLIDE
\cscsthankyou{Thank you for your attention.}

\end{document}
