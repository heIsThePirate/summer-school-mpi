\documentclass[aspectratio=43]{beamer}
\usepackage{ragged2e}
\usepackage{multirow}
\usepackage{alltt}

\usetheme{CSCS}

\newcommand{\footlinetext}{Summer School 2015 - MPI}

% Select the image for the title page
%\newcommand{\picturetitle}{cscs_images/image3.pdf}
%\newcommand{\picturetitle}{cscs_images/image5.pdf}
\newcommand{\picturetitle}{cscs_images/image6.pdf}

\author{Maxime Martinasso, CSCS}
\title{Message Passing Interface (MPI)}
\subtitle{Summer School 2015 - Effective High Performance Computing}
\date{\today}

\begin{document}

% TITLE SLIDE
\cscstitle

\begin{frame}{Previous course summary}
\begin{itemize}
\item Point-to-point communication, blocking and non-blocking
\item Collective operations
\item Communicators and groups
\item Cartesian and Graph topology
\end{itemize}
\end{frame}

\begin{frame}{Course Objectives}
\begin{itemize}
\item Construct and use MPI derived datatypes
\end{itemize}
\end{frame}

% TABLE OF CONTENT SLIDE
\cscstableofcontents[hideallsubsections]{General Course Structure}

\section{An introduction to MPI}
\section{Point-to-point communications}
\section{Collective communications}
\section{Topology}
\section{Datatypes}
\cscstableofcontents[currentsection]{General Course Structure}

% CHAPTER SLIDE
\cscschapter{MPI derived datatypes}

\subsection{Construct datatype}

\begin{frame}[fragile]{Using MPI derived datatypes}
MPI derived datatypes (differently from C or Fortran) are created (and destroyed) at run-time through calls to MPI library routines.\\
Implementation steps:
\begin{enumerate}
\item Construct the datatype;
\item Allocate the datatype;
\item Use the datatype;
\item Deallocate the datatype.
\end{enumerate}
\end{frame}

\begin{frame}[fragile]{Construct a datatype}
\begin{itemize}
\item \verb+MPI_Type_contiguous+:\\
Produces a new datatype by making count copies of an existing data type.

\item \verb+MPI_Type_vector, MPI_Type_hvector+:\\
Similar to contiguous, but allows for regular gaps (stride) in the displacements.
\verb+MPI_Type_hvector+ is identical to \verb+MPI_Type_vector+ except that stride is specified in bytes.

\item \verb+MPI_Type_indexed, MPI_Type_hindexed+:\\
An array of displacements of the input data type is provided as the map for the new data type.
\verb+MPI_Type_hindexed+ is identical to \verb+MPI_Type_indexed+ except that offsets are specified in bytes.

\item \verb+MPI_Type_struct+:\\
The most general of all derived datatypes. The new data type is formed according to completely defined map of the component data types.
\end{itemize}

\end{frame}

\begin{frame}[fragile]{Allocate and destroy the Datatype}
A constructed datatype must be committed to the system before it can be used in a communication.\\
\footnotesize
C:
\vspace{-0.2cm}
\begin{verbatim}
int MPI_Type_commit(MPI_datatype *datatype)
int MPI_Type_free(MPI_datatype *datatype)
\end{verbatim}
\vspace{-0.2cm}
Fortran:
\vspace{-0.2cm}
\begin{verbatim}
MPI_TYPE_COMMIT(DATATYPE, IERR)
MPI_TYPE_FREE(DATATYPE, IERR)
\end{verbatim}
\end{frame}

\begin{frame}[fragile]{Contiguous Datatype}
\verb+MPI_TYPE_CONTIGOUS+ constructs a typemap consisting of the replication of a datatype into contiguous locations.
\footnotesize
\begin{verbatim}
MPI_TYPE_CONTIGUOUS(count, oldtype, newtype, ierr)
\end{verbatim}
\vspace{-0.2cm}
\begin{black1block}{}
\begin{tabular}{rp{8cm}}
\textbf{count} & number of BLOCKs to be added\\
\textbf{oldtype} & oldtype Datatype of each element\\
\textbf{newtype} & new derived datatype\\
\end{tabular}
\end{black1block}
REMEMBER: BLOCK=contiguous elements of the same type.
\end{frame}

\begin{frame}[fragile]{Contiguous Datatype: example}
\begin{center}
\begin{tabular}{|c|c|c|c|}
    \multicolumn{4}{c}{array $a[][]$=}\\\hline
0.0  & 0.1  & 0.2  & 0.3\\\hline
0.4  & 0.5  & 0.6  & 0.7\\\hline
0.8  & 0.9  & 0.10 & 0.11\\\hline
0.12 & 0.13 & 0.14 & 0.15\\\hline
\end{tabular}
\end{center}
Create a new type of 4 floats representing a row in $a$.\\
\verb+MPI_Type_contiguous(4, MPI_FLOAT, &MyRowType)+\\[0.5cm]
Use the new type to send one row:\\
\verb+MPI_Send(&a[2][0], 1, MyRowType, dest, tag, comm)+\\[0.5cm]
Data sent is:
\begin{tabular}{|c|c|c|c|}
\hline
\color{cscsblue}0.8  & \color{cscsblue}0.9  & \color{cscsblue}0.10 & \color{cscsblue}0.11\\
\hline
\end{tabular}

\end{frame}

\subsection{Contiguous datatype}
\begin{frame}[fragile]{Contiguous Datatype with stride}
\verb+MPI_TYPE_CONTIGOUS+ constructs a typemap consisting of the replication of a datatype into contiguous locations.
\footnotesize
\begin{verbatim}
MPI_TYPE_VECTOR(count, blocklength, stride, oldtype, newtype, ierr)
\end{verbatim}
\vspace{-0.2cm}
\begin{black1block}{}
\begin{tabular}{rp{8cm}}
\textbf{count} & number of BLOCKs to be added\\
\textbf{blocklength} & Number of elements in block\\
\textbf{stride} & Number of elements (NOT bytes) between start of each block\\
\textbf{oldtype} & oldtype Datatype of each element\\
\textbf{newtype} & new derived datatype\\
\end{tabular}
\end{black1block}
The Vector constructor is similar to contiguous, but allows for regular gaps or overlaps (stride) in the displacements.
\end{frame}

\begin{frame}[fragile]{Contiguous Datatype with stride: example}
\begin{center}
\begin{tabular}{|c|c|c|c|}
    \multicolumn{4}{c}{array $a[][]$=}\\\hline
0.0  & 0.1  & 0.2  & 0.3\\\hline
0.4  & 0.5  & 0.6  & 0.7\\\hline
0.8  & 0.9  & 0.10 & 0.11\\\hline
0.12 & 0.13 & 0.14 & 0.15\\\hline
\end{tabular}
\end{center}
Create a new type of 4 floats representing a col in $a$.\\
\verb+count = 4; blocklength=1; stride = 4;+
\verb+MPI_Type_contiguous(count, blocklength, stride,+
\verb+                    MPI_FLOAT, &MyColType)+\\[0.5cm]
Use the new type to send one row:\\
\verb+MPI_Send(&a[0][2], 1, MyColType, dest, tag, comm)+\\[0.5cm]
Data sent is:
\begin{tabular}{|c|c|c|c|}
\hline
\color{cscsblue}0.2  & \color{cscsblue}0.6  & \color{cscsblue}0.10 & \color{cscsblue}0.14\\
\hline
\end{tabular}

\end{frame}

\subsection{Indexed datatype}
\begin{frame}[fragile]{Indexed Datatype}
\verb+MPI_TYPE_INDEXED+ constructs a typemap consisting of the replication of a datatype from locations defined by an array of block lengths and an array of displacements.
\footnotesize
\begin{verbatim}
MPI_TYPE_INDEXED(count, array_blocklengths, array_displacements,
                 oldtype, newtype, ierr)
\end{verbatim}
\vspace{-0.2cm}
\begin{black1block}{}
\begin{tabular}{rp{6.5cm}}
\textbf{count} & number of BLOCKs to be added and number of elements in the following arrays\\
\textbf{array\_blocklengths} & number of instances of oldtype in each block\\
\textbf{array\_displacements} & displacement of each block in units of extent (oldtype)\\
\textbf{oldtype} & oldtype Datatype of each element\\
\textbf{newtype} & new derived datatype\\
\end{tabular}
\end{black1block}
\end{frame}

\begin{frame}[fragile]{Indexed Datatype: example}
count = 3;\\
oldtype = MPI\_INT;\\
array\_blocklengths=
\begin{tabular}{|c|c|c|}
\hline
2  & 3  & 1 \\
\hline
\end{tabular}\\
array\_displacements=
\begin{tabular}{|c|c|c|}
\hline
0  & 3  & 9 \\
\hline
\end{tabular}\\[0.5cm]

Selected blocks are:
\begin{tabular}{|c|c|c|c|c|c|c|c|c|c|c|}
\hline
\color{cscsred}0  & \color{cscsred}1  & 2 & \color{cscsred}3 & \color{cscsred}4 & \color{cscsred}5 & 6 & 7 & 8 & \color{cscsred}9 & 10\\
\hline
\end{tabular}
\end{frame}


\subsection{Struct datatype}
\begin{frame}[fragile]{Struct Datatype}
\verb+MPI_TYPE_STRUCT+ constructs a typemap consisting of different datatype from locations defined by an array of block lengths and an array of displacements.
Displacements are expressed in bytes (since the type can change!!!).
\footnotesize
\begin{verbatim}
MPI_TYPE_STRUCT(count, array_blocklengths, array_displacements,
                 array_oldtype, newtype, ierr)
\end{verbatim}
\vspace{-0.2cm}
\begin{black1block}{}
\begin{tabular}{rp{6.5cm}}
\textbf{count} & number of BLOCKs to be added and number of elements in the following arrays\\
\textbf{array\_blocklengths} & number of instances of oldtype in each block\\
\textbf{array\_displacements} & displacement in BYTES of each block\\
\textbf{array\_oldtype} & oldtype Datatype of each element\\
\textbf{newtype} & new derived datatype\\
\end{tabular}
\end{black1block}
\end{frame}

\begin{frame}[fragile]{Struct Datatype: example}
count = 3;\\
oldtype = MPI\_INT;\\
array\_blocklengths=
\begin{tabular}{|c|c|c|}
\hline
2  & 2  & 1 \\
\hline
\end{tabular}\\
array\_displacements (in bytes)=
\begin{tabular}{|c|c|c|}
\hline
0  & 12  & 36 \\
\hline
\end{tabular}\\
array\_oldtypes=
\begin{tabular}{|c|c|c|}
\hline
\verb+MPI_INT+  & \verb+MPI_DOUBLE+  & \verb+MPI_FLOAT+ \\
\hline
\end{tabular}\\[0.5cm]

A block is 4 Bytes long.\\
Selected blocks are:
\begin{tabular}{|c|c|c|c|c|c|c|c|c|c|c|}
\hline
\color{cscsred}0  & \color{cscsred}1  & 2 & \color{cscsred}3 & \color{cscsred}4 & \color{cscsred}5 & \color{cscsred}6 & 7 & 8 & \color{cscsred}9 & 10\\
\hline
\end{tabular}\\


\end{frame}

\subsection{Sub-array datatype}
\begin{frame}[fragile]{Sub-array Datatype}
\justifying
The subarray type constructor creates an MPI datatype describing an n-dimensional subarray of an n-dimensional array. The subarray may be placed anywhere within the full array.
\footnotesize
\begin{verbatim}
MPI_TYPE_CREATE_SUBARRAY(ndims, array_sizes, array_subsizes,
                         array_starts, order, oldtype, newtype)
\end{verbatim}
\vspace{-0.2cm}
\begin{black1block}{}
\begin{tabular}{rp{6.5cm}}
\textbf{ndims} & number of array dimensions\\
\textbf{array\_sizes} & number of elements of type oldtype in each dimension of the full array\\
\textbf{array\_subsizes} & number of elements of type oldtype in each dimension of the subarray\\
\textbf{order} & array storage order flag\\
\textbf{array\_starts} & starting coordinates of the subarray in each dimension\\
\textbf{oldtype} & oldtype Datatype of each element\\
\textbf{newtype} & new derived datatype\\
\end{tabular}
\end{black1block}
\end{frame}

\begin{frame}[fragile]{Sub-array Datatype: example}
A 100x100 2D array of double precision floating point numbers distributed among 4 processes such that each process has a block of 25 columns.

\footnotesize
\begin{verbatim}
double subarray[100][25];
MPI_Datatype filetype;
int sizes[2], subsizes[2], starts[2];
int rank;

MPI_Comm_rank(MPI_COMM_WORLD, &rank);

sizes[0]=100; sizes[1]=100;
subsizes[0]=100; subsizes[1]=25;
starts[0]=0; starts[1]=rank*25;

MPI_Type_create_subarray(2, sizes, subsizes, starts, MPI_ORDER_C,
                         MPI_DOUBLE, &filetype);
\end{verbatim}

\end{frame}

\begin{frame}[fragile]{Other functions}
\begin{itemize}
    \item Manage types:\\\hspace{1cm}\verb+MPI_Comm_extent, MPI_Type_dup+\ldots
    \item Getter for types:\\\hspace{1cm}\verb+MPI_Type_size, MPI_Type_get_contents+\ldots
\end{itemize}
\end{frame}


\begin{frame}{Practicals}
    \begin{brown2block}{Exercise 6}
    \end{brown2block}
\end{frame}



\section{Parallel I/O with MPI}

% THANK YOU SLIDE
\cscsthankyou{Thank you for your attention.}

\end{document}
